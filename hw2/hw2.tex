\documentclass{article}

\usepackage{fancyhdr}
\usepackage{extramarks}
\usepackage{amsmath}
\usepackage{amsthm}
\usepackage{amsfonts}
\usepackage{tikz}
\usepackage[plain]{algorithm}
\usepackage{algpseudocode}
\usepackage{graphicx}
\usepackage{gensymb}
\usepackage[framed,numbered,autolinebreaks,useliterate]{mcode}

\graphicspath{{./images/}}

\usetikzlibrary{automata,positioning}

%
% Basic Document Settings
%

\topmargin=-0.45in
\evensidemargin=0in
\oddsidemargin=0in
\textwidth=6.5in
\textheight=9.0in
\headsep=0.25in

\linespread{1.1}

\pagestyle{fancy}
\lhead{\hmwkAuthorName}
\chead{\hmwkClass\ \hmwkTitle}
\rhead{\firstxmark}
\lfoot{\lastxmark}
\cfoot{\thepage}

\renewcommand\headrulewidth{0.4pt}
\renewcommand\footrulewidth{0.4pt}

\setlength\parindent{0pt}

%
% Create Problem Sections
%

\newcommand{\enterProblemHeader}[1]{
    \nobreak\extramarks{}{Problem {#1} continued on next page\ldots}\nobreak{}
    \nobreak\extramarks{Problem {#1} (continued)}{Problem {#1} continued on next page\ldots}\nobreak{}
}

\newcommand{\exitProblemHeader}[1]{
    \nobreak\extramarks{Problem {#1} (continued)}{Problem {#1} continued on next page\ldots}\nobreak{}
    % \stepcounter{#1}
    \nobreak\extramarks{Problem {#1}}{}\nobreak{}
}

\setcounter{secnumdepth}{0}
\newcounter{partCounter}

\newcommand{\problemNumber}{0.0}

\newenvironment{homeworkProblem}[1][-1]{
    \renewcommand{\problemNumber}{{#1}}
    \section{Problem \problemNumber}
    \setcounter{partCounter}{1}
    \enterProblemHeader{\problemNumber}
}{
    \exitProblemHeader{\problemNumber}
}

%
% Homework Details
%   - Title
%   - Class
%   - Author
%

\newcommand{\hmwkTitle}{Homework\ \#2}
\newcommand{\hmwkClass}{RBE 500}
\newcommand{\hmwkAuthorName}{\textbf{Arjan Gupta}}

%
% Title Page
%

\title{
    \vspace{2in}
    \textmd{\textbf{\hmwkClass\ \hmwkTitle}}\\
    \vspace{3in}
}

\author{\hmwkAuthorName}
\date{}

\renewcommand{\part}[1]{\textbf{\large Part \Alph{partCounter}}\stepcounter{partCounter}\\}

%
% Various Helper Commands
%

% Useful for algorithms
\newcommand{\alg}[1]{\textsc{\bfseries \footnotesize #1}}

% For derivatives
\newcommand{\deriv}[1]{\frac{\mathrm{d}}{\mathrm{d}x} (#1)}

% For partial derivatives
\newcommand{\pderiv}[2]{\frac{\partial}{\partial #1} (#2)}

% Integral dx
\newcommand{\dx}{\mathrm{d}x}

% Alias for the Solution section header
\newcommand{\solution}{\textbf{\large Solution}}

% Probability commands: Expectation, Variance, Covariance, Bias
\newcommand{\E}{\mathrm{E}}
\newcommand{\Var}{\mathrm{Var}}
\newcommand{\Cov}{\mathrm{Cov}}
\newcommand{\Bias}{\mathrm{Bias}}

\begin{document}

\maketitle

\nobreak\extramarks{Problem 3.5}{}\nobreak{}

\pagebreak

\begin{homeworkProblem}[3.5]
    Consider the three-link articulated robot of Figure 3.16. Derive the forward kinematic equations using the DH convention.
    \begin{figure}[h]
        \includegraphics[scale=0.25]{figure3-16.png}
        \centering
    \end{figure}

    \textbf{Solution}
    \vspace{0.1in}\\
    First we assign coordinate frames 0 through 3 (links 0 through 3). This is done as per the following figure.
    \begin{figure}[h]
        \includegraphics[scale=0.4]{problem-3-5-figure-frames.png}
        \centering
    \end{figure}

    Now, we create a table for quantities \(\alpha_i, a_i, \theta_i, d_i\) for links 1 through 3.

    \begin{table}[h!]
        \begin{center}
            \begin{tabular}{|c|c|c|c|c|}
            \hline
            Link & $\alpha_i$ & $a_i$ & $\theta_i$ & $d_i$ \\
            \hline
            1 & -90\degree & 0 & $\theta_1$ & $d_1$ \\
            2 & 0 & $a_2$ & $\theta_2$ & 0\\
            3 & 0 & $a_3$ & $\theta_3$ & 0\\
            \hline
            \end{tabular}
        \end{center}
    \end{table}

    Next, we use the matrix obtained from equation 3.10 of the textbook to calculate \(A_1, A_2, A_3\).
    
    \[
        A_1 =
        \begin{bmatrix}
            \cos\theta_1 & -\sin\theta_1\cos(-90\degree) & \sin\theta_1\sin(-90\degree) & 0\cdot\cos\theta_1\\
            \sin\theta_1 & \cos\theta_1\cos(-90\degree) & -\cos\theta_1\sin(-90\degree) & 0\cdot\sin\theta_1\\
            0 & \sin(-90\degree) & \cos(-90\degree) & d_1\\
            0 & 0 & 0 & 1
        \end{bmatrix}
        =
        \begin{bmatrix}
            c_1 & 0 & -s_1 & 0\\
            s_1 & 0 & c_1 & 0\\
            0 & -1 & 0 & d_1\\
            0 & 0 & 0 & 1
        \end{bmatrix}
    \]
    
    \vspace{0.2in}
    Where \(s_1 = \sin\theta_1\) and \(c_1 = \cos\theta_1\). Similarly,

    \[
        A_2 =
        \begin{bmatrix}
            c_2 & -s_2 & 0 & a_2c_2\\
            s_2 & c_2 & 0 & a_2s_2\\
            0 & 0 & 1 & 0\\
            0 & 0 & 0 & 1
        \end{bmatrix},
        A_3 =
        \begin{bmatrix}
            c_3 & -s_3 & 0 & a_3c_3\\
            s_3 & c_3 & 0 & a_3s_3\\
            0 & 0 & 1 & 0\\
            0 & 0 & 0 & 1
        \end{bmatrix}
    \]

    \vspace{0.2in}
    Now we can find \(T_3^0 = A_1A_2A_3\). We use the following MATLAB code to compute this.

    \lstinputlisting{./prob3_5.m}

    \vspace{0.2in}
    Therefore,
    \[
        T_3^0 =
        \begin{bmatrix}
            c_{1}\,c_{2}\,c_{3}-c_{1}\,s_{2}\,s_{3} & -c_{1}\,c_{2}\,s_{3}-c_{1}\,c_{3}\,s_{2} & -s_{1} & d_{1}+a_{2}\,c_{1}\,c_{2}-a_{3}\,c_{1}\,s_{2}\,s_{3}+a_{3}\,c_{1}\,c_{2}\,c_{3}\\
            c_{2}\,c_{3}\,s_{1}-s_{1}\,s_{2}\,s_{3} & -c_{2}\,s_{1}\,s_{3}-c_{3}\,s_{1}\,s_{2} & c_{1} & a_{2}\,c_{2}\,s_{1}-a_{3}\,s_{1}\,s_{2}\,s_{3}+a_{3}\,c_{2}\,c_{3}\,s_{1}\\
            -c_{2}\,s_{3}-c_{3}\,s_{2} & s_{2}\,s_{3}-c_{2}\,c_{3} & 0 & d_{1}-a_{2}\,s_{2}-a_{3}\,c_{2}\,s_{3}-a_{3}\,c_{3}\,s_{2}\\
            -c_{2}\,s_{3}-c_{3}\,s_{2} & s_{2}\,s_{3}-c_{2}\,c_{3} & 0 & d_{1}-a_{2}\,s_{2}-a_{3}\,c_{2}\,s_{3}-a_{3}\,c_{3}\,s_{2}\\
            0 & 0 & 0 & 1
        \end{bmatrix}
    \]
    
    \vspace{0.2in}
    This gives the configuration of frame 3 with respect to the base frame (frame 0).

\end{homeworkProblem}

\nobreak\extramarks{Problem 3.6}{}\nobreak{}

\pagebreak

\begin{homeworkProblem}[3.6]

    Consider the three-link Cartesian manipulator of Figure 3.17. Derive the forward kinematic equations using the DH convention.
    \begin{figure}[h]
        \includegraphics[scale=0.5]{figure-3.17.png}
        \centering
    \end{figure}

    \textbf{Solution}
    \vspace{0.1in}\\
    First we assign coordinate frames 0 through 3 (links 0 through 3). This is done as per the following figure.
    \begin{figure}[h]
        \includegraphics[scale=0.35]{figure3-17-frames.png}
        \centering
    \end{figure}

    \vspace{0.2in}

    Now, we create a table for quantities \(\alpha_i, a_i, \theta_i, d_i\) for links 1 through 3.

    \vspace{0.2in}

    \begin{table}[h!]
        \begin{center}
            \begin{tabular}{|c|c|c|c|c|}
            \hline
            Link & $\alpha_i$ & $a_i$ & $\theta_i$ & $d_i$ \\
            \hline
            1 & 90\degree & 0 & 0 & $d_1$ \\
            2 & 90\degree & 0 & 90\degree & $d_2$\\
            3 & 0 & 0 & 0 & $d_3$\\
            \hline
            \end{tabular}
        \end{center}
    \end{table}

    \vspace{0.1in}

    Next, we use the matrix obtained from equation 3.10 of the textbook to calculate \(A_1, A_2, A_3\).

    \[
        A_1 =
        \begin{bmatrix}
            1 & 0 & 0 & 0\\
            0 & 0 & -1 & 0\\
            0 & 1 & 0 & d_1\\
            0 & 0 & 0 & 1
        \end{bmatrix},
        A_2 =
        \begin{bmatrix}
            0 & 0 & 1 & 0\\
            1 & 0 & 0 & 0\\
            0 & 1 & 0 & d_2\\
            0 & 0 & 0 & 1
        \end{bmatrix},
        A_3 =
        \begin{bmatrix}
            1 & 0 & 0 & 0\\
            0 & 1 & 0 & 0\\
            0 & 0 & 1 & d_3\\
            0 & 0 & 0 & 1
        \end{bmatrix}
    \]

    \vspace{0.2in}
    Now we can find \(T_3^0 = A_1A_2A_3\). We use the following MATLAB code to compute this.

    \lstinputlisting{./prob3_6.m}

    \vspace{0.2in}
    Therefore,

    \[
        T_3^0 =
        \begin{bmatrix}
            0 & 0 & 1 & d_{3}\\
            0 & -1 & 0 & -d_{2}\\
            1 & 0 & 0 & d_{1}\\
            0 & 0 & 0 & 1
        \end{bmatrix}
    \]

    \vspace{0.2in}
    This gives the configuration of frame 3 with respect to the base frame (frame 0).
    
\end{homeworkProblem}

\nobreak\extramarks{Problem 5.3}{}\nobreak{}

\pagebreak

\begin{homeworkProblem}[5.3]

    Solve the inverse position kinematics for the cylindrical manipulator of Figure 5.15.
    \begin{figure}[h]
        \includegraphics[scale=0.3]{figure-5.3.png}
        \centering
    \end{figure}

    \textbf{Solution}
    \vspace{0.1in}\\
    Let us draw the base frame's axes $x_0 y_0 z_0$ as shown in the figure below. Also, let us
    select a point $(x_c, y_c, z_c)$ as the wrist center at the far end of the second prismatic joint, as shown.
    \begin{figure}[h]
        \includegraphics[scale=0.45]{figure-5.3-sketched.png}
        \centering
    \end{figure}

    \vspace{0.2in}
    To solve the inverse position kinematics problem for this configuration, we need to find $q_1, q_2, q_3$, or more precisely, $\theta_1, d_2, d_3$.

    \vspace{0.2in}
    Using the $Atan2()$ algorithmic function as described in the appendix of the textbook, we determine from the figure that,

    \[
        \boxed{\theta_1 = Atan2(x_c, y_c)}
    \]
        
    or, alternatively,
        
    \[
        \boxed{\theta_1 = \pi + Atan2(x_c, y_c)}
    \]

    \vspace{0.2in}
    Furthermore, it is apparent that

    \[
        \begin{split}
            z_c = 1 + d_2\\
            \boxed{d_2 = z_c - 1}
        \end{split}
    \]

    We also see from the figure that

    \[
        r = \sqrt[]{{x_c}^2 + {y_c}^2}
    \]

    But,

    \[
        r = 1 + d_3
    \]

    So,

    \[
        \boxed{d_3 = \sqrt[]{{x_c}^2 + {y_c}^2} - 1}
    \]

    \vspace{0.2in}
    This solves the inverse position kinematics problem for the given cylindrical configuration.

\end{homeworkProblem}

\nobreak\extramarks{Problem 5.5}{}\nobreak{}

\pagebreak

\begin{homeworkProblem}[5.5]
    Add a spherical wrist to the three-link cylindrical arm of Problem 5--3 and write the complete inverse kinematics solution.
    \vspace{0.2in}

    \textbf{Solution}
    \vspace{0.1in}\\
    Let us consider a spherical wrist identical to the one used in the textbook. We attach this spherical wrist such that the wrist center,
    now denoted by vector $o_c$, coincides with the point $(x_c,y_c,z_c)$ as we found in Problem 5--3. We have concluded that the wrist center
    lies at this point because axes \(z_3, z_4, z_5\) intersect at this point. This point is also where the origins \(o_3, o_4, o_5\) lie as 
    per the frame assignment by DH conventions. We also know that the position of $o_c$ does not change despite \(\theta_4, \theta_5, \theta_6\)
    being variables.\\
    
    Also, for the sake of clearly denoting $d_3$ as joint variable $q_3$, we have now used $u_3$ in the figure. It is still the same distance 
    found in Problem 5--3, i.e. \(u_3 = \sqrt[]{{x_c}^2 + {y_c}^2} - 1\). Given our placement of frame 2, we now have $d_3 = u_3 + 1$. Therefore,
    \(q_3 = d_3 = \sqrt[]{{x_c}^2 + {y_c}^2}\).\\

    An additional thing to note is that although $z_6$ is along the same direction as $z_5$, the coordinates of $o_6$ lie on the point shown by
    the vector $o_6$.
    \begin{figure}[h]
        \includegraphics[scale=0.45]{figure-5.5-sketched.png}
        \centering
    \end{figure}

    Before we proceed further, let us make a brief list of steps we need to take to solve the complete inverse kinematics problem for our 
    particular manipulator's configuration.
    \begin{enumerate}
        \item Find wrist center $o_c$.
        \item Find $q1, q2, q3$.
        \item Perform forward kinematics to arrive at \(R^0_3 = {(R^3_0)}^T\).
        \item Get \(R^3_6 = R^3_0R^0_6\).
        \item Use $R^3_6$ to find $\phi, \theta, \psi$ of Euler configuration to find $q_4, q_5, q_6$.
    \end{enumerate}
    In essence, once we have found all joint variables given the end-effector's homogeneous transformation, we have solved the inverse kinematics problem.\\

    \textbf{Step 1}

    The end-effector's homogeneous transformation is known to us as the $4\times4$ matrix

    \[
        H^0_6 =
        \begin{bmatrix}
             R^0_6 & o^0_6\\
             0 & 1
        \end{bmatrix}
    \]

    where

    \[
        R^0_6 = 
        \begin{bmatrix}
            r_{11} & r_{12} & r_{13}\\
            r_{21} & r_{22} & r_{23}\\
            r_{31} & r_{32} & r_{33}
        \end{bmatrix},
        o^0_6 =
        \begin{bmatrix}
            x_6 \\
            y_6 \\
            z_6
        \end{bmatrix}
    \]

    Where $o^0_6$ is $o_6$ as shown in the diagram. As shown in the figure, we can establish a relationship between $o_6$ and $o_c$ as

    \[
        \begin{split}
            o_c = o_6 - d_6R^0_6
            \begin{bmatrix}
                0\\
                0\\
                1
            \end{bmatrix}\\
            \begin{bmatrix}
                x_c \\
                y_c \\
                z_c
            \end{bmatrix}
            =
            \begin{bmatrix}
                x_6 - d_6r_{13} \\
                y_6 - d_6r_{23} \\
                z_6 - d_6r_{33}
            \end{bmatrix}
        \end{split}
    \]

    

    Where $d_6$ is a scalar.
    \vspace{0.15in}

    \textbf{Step 2}
    
    We have already found $q_1, q_2, q_3$ in Problem 5.3. We summarize our findings here,

    \[
        q_1 = \theta_1 = Atan2(x_c, y_c)\\
    \]
    \[
        q_2 = d_2 = z_c - 1\\
    \]
    \[
        q_3 = d_3 = \sqrt[]{{x_c}^2 + {y_c}^2}
    \]

    We have discarded the second possibility of $q_1$ as our choice.

    \vspace{0.15in}

    \textbf{Step 3}

    We perform forward kinematics for the first three joint variables. Here is our table,

    \begin{table}[h!]
        \begin{center}
            \begin{tabular}{|c|c|c|c|c|}
            \hline
            Link & $\alpha_i$ & $a_i$ & $\theta_i$ & $d_i$ \\
            \hline
            1 & 0 & 0 & $\theta_1$ & 1 \\
            2 & 90\degree & 0 & 0 & $d_2$ \\
            3 & 0 & 0 & 0 & $d_3$\\
            \hline
            \end{tabular}
        \end{center}
    \end{table}
    
    Next, we use the matrix obtained from equation 3.10 of the textbook to calculate \(A_1, A_2, A_3\).

    \[
        A_1 =
        \begin{bmatrix}\cos\left(\theta _{1}\right) & -\sin\left(\theta _{1}\right) & 0 & 0\\ \sin\left(\theta _{1}\right) & \cos\left(\theta _{1}\right) & 0 & 0\\ 0 & 0 & 1 & 1\\ 0 & 0 & 0 & 1 \end{bmatrix}
        ,
        A_2 = \begin{bmatrix} 1 & 0 & 0 & 0\\ 0 & 0 & -1 & 0\\ 0 & 1 & 0 & d_{2}\\ 0 & 0 & 0 & 1 \end{bmatrix}
        ,
        A_3 = \begin{bmatrix} 1 & 0 & 0 & 0\\ 0 & 1 & 0 & 0\\ 0 & 0 & 1 & d_{3}\\ 0 & 0 & 0 & 1 \end{bmatrix}
    \]

    We obtain $T^0_3$ using the following MATLAB code.

    \lstinputlisting{./prob5_5_fk.m}

    \vspace{0.2in}
    Therefore,
    \[
        T^0_3 =
        \begin{bmatrix} c_{1} & 0 & s_{1} & d_{3}\,s_{1}\\ s_{1} & 0 & -c_{1} & -c_{1}\,d_{3}\\ 0 & 1 & 0 & d_{2}+1\\ 0 & 0 & 0 & 1 \end{bmatrix}
    \]

    From here, it is clear that

    \[
        R^0_3 =
        \begin{bmatrix}
            c_{1} & 0 & s_{1} \\ s_{1} & 0 & -c_{1} \\ 0 & 1 & 0
        \end{bmatrix}
    \]

    \vspace{0.15in}

    \textbf{Step 4}

    We know that $R^3_0 = {(R^0_3)}^T$. Given this fact, we use the following MATLAB code to calculate $R^3_6$.

    \lstinputlisting{./prob5_5_step4.m}

    \[
        R^3_6 =
        \begin{bmatrix}
            c_{1}\,r_{11}+r_{21}\,s_{1} & c_{1}\,r_{12}+r_{22}\,s_{1} & c_{1}\,r_{13}+r_{23}\,s_{1}\\
            r_{31} & r_{32} & r_{33}\\
            r_{11}\,s_{1}-c_{1}\,r_{21} & r_{12}\,s_{1}-c_{1}\,r_{22} & r_{13}\,s_{1}-c_{1}\,r_{23}
        \end{bmatrix}
    \]
    
    \vspace{0.15in}
    \textbf{Step 5}

    \vspace{0.15in}
    For the final step, we make use of the Euler angles matrix, where $q_4 = \phi, q_5 = \theta, q_6 = \psi$. The matrix for this, as given in
    the textbook, is

    \[
        R^3_6 =
        \begin{bmatrix}
            c_4c_5c_6 - s_4s_6 & -c_4c_5s_6 - s_4c_6 & c_4s_5\\
            s_4c_5c_6 + c_4s_6 & -s_4c_5s_6 + c_4c_6 & s_4s_5\\
            -s_5c_6 & s_5s_6 & c_5
        \end{bmatrix}
    \]

    Now, if we equate this with our matrix from Step 4, we get the third column as,
    \[
        \begin{split}
            c_4s_5 = c_{1}\,r_{13}+r_{23}\,s_{1}\\
            s_4s_5 = r_{33}\\
            c_5 = r_{13}\,s_{1}-c_{1}\,r_{23}
        \end{split}
    \]
    And the third row as,
    \[
        \begin{split}
            -s_5c_6 = r_{11}\,s_{1}-c_{1}\,r_{21}\\
            s_5s_6 = r_{12}\,s_{1}-c_{1}\,r_{22}\\
            c_5 = r_{13}\,s_{1}-c_{1}\,r_{23}
        \end{split}
    \]

    So, finally, using equations (2.29), (2.30), (2.31), and (2.32) of the textbook,

    \[
        \boxed{\theta_5 = Atan2(r_{13}\,s_{1}-c_{1}\,r_{23}, \sqrt[]{1 - {(r_{13}\,s_{1}-c_{1}\,r_{23})}^2})}\\
    \]
    or
    \[
        \boxed{\theta_5 = Atan2(r_{13}\,s_{1}-c_{1}\,r_{23}, -\sqrt[]{1 - {(r_{13}\,s_{1}-c_{1}\,r_{23})}^2})}\\
    \]

    where $Atan2$ is the two-argument algorithmic arctangent function defined in Appendix A of the textbook.\\

    Also,
    \[
        \begin{split}
            \boxed{\theta_4 = Atan2(c_{1}\,r_{13}+r_{23}\,s_{1}, r_{33})} \\
            \boxed{\theta_6 = Atan2(-r_{11}\,s_{1}+c_{1}\,r_{21}, r_{12}\,s_{1}-c_{1}\,r_{22})}\\
        \end{split}
    \]

\end{homeworkProblem}

\nobreak\extramarks{ROS2 Report}{}\nobreak{}

\pagebreak

\section{Report for ROS2 Portion}

    For our ROS part of this assignment, we created a subscriber just like how we did in the last homework assignment. 
    We did not create a publisher, however instead we tested the subscriber's conversion of data by using the "ros2 topic pub ..."
    command. A sample of this has been submitted under publish.sh.\\
    
    The subscriber listens for a Float32MultiArray, and when it is received, it checks if the array length is 3. If it is not
    3, then the subscriber notifies that no action will be taken. When the array length is exactly 3, the subscriber passes the array
    to a helper function to convert the incoming euler angle data to quaternions. The euler angle order is assumed to be given
    as [yaw, pitch, roll]. We use the formula given on the Wikipedia page to convert the Euler angles to quaternions. The converted
    data is then printed to the terminal. A screenshot of this output has been submitted as well. 

\end{document}