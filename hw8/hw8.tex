\documentclass[12pt]{article}
\usepackage{lipsum} %This package just generates Lorem Ipsum filler text. 
\usepackage{fancyhdr}
\usepackage[framed,numbered,autolinebreaks,useliterate]{mcode}
\usepackage{listings}
\usepackage{amsmath}
\usepackage{amsthm}
\usepackage{amsfonts}
\usepackage{tikz}
\usepackage[plain]{algorithm}

\newcommand{\hmwkTitle}{Homework\ \#8}
\newcommand{\hmwkClass}{RBE 500}
\newcommand{\hmwkAuthorName}{\textbf{Arjan Gupta}}

% Document settings

\topmargin=-0.45in
\evensidemargin=0in
\oddsidemargin=0in
\textwidth=6.5in
\textheight=9.0in
\headsep=0.25in

\linespread{1.1}

\pagestyle{fancy}
\lhead{\hmwkAuthorName}
\chead{\hmwkClass\ \hmwkTitle}
\cfoot{\thepage}

\renewcommand\headrulewidth{0.4pt}
\renewcommand\footrulewidth{0.4pt}

\setlength\parindent{0pt}
\setlength{\headheight}{15pt}


% Title Page
\title{
    \vspace{2in}
    \textmd{\textbf{\hmwkClass\ \hmwkTitle}}\\
    \vspace{3in}
}
\author{\hmwkAuthorName}
\date{}

% Function for citations
\newcommand{\supercite}[1]{~{\textsuperscript{\cite{#1}}}}

\begin{document}

\maketitle

\pagebreak

\begin{center}
    \large\textbf{Bayes Filter Report}
\end{center}
I started my Bayes Filter implementation in a python file called
\lstinline{hw8_bayes_filter.py}. At the top of the file, I added a \lstinline{VERBOSE}
flag that helps me toggle on and off output that is helpful for debugging, but
probably superfluous for the grader. I defined a list of states of the door,
which are \textit{open} and \textit{closed}. I also defined the list of 
`base' beliefs, which
are given in the slides/textbook as the following.
\begin{align*}
    bel(X_0 = \mathbf{open})\;\; &=\;\; 0.5\\
    bel(X_0 = \mathbf{closed})\;\; &=\;\; 0.5
\end{align*}
These are the initial beliefs (as denoted by $X_0$), which means that these values
will adjust as the script runs through the iteration cases. In my code, these are
simply a python list, followed by two helper functions. One helper function helps
us print the current belief values.
The other helper function makes sure our states, which are strings, are `mapped'
to the belief values, and returns them as such.
\lstinputlisting[language=python, firstline=8, lastline=25]{hw8_bayes_filter.py}
\vspace{0.15in}
Next, I started defining the helper function to get the belief values associated
with the sensor, i.e.\ the measurement beliefs. In the slides, these are given
as the following.
\begin{align*}
    p(Z_t = \mathbf{sense\_open} \;|\; X_t = \mathbf{is\_open})\;\;&=\;\;0.6\\
    p(Z_t = \mathbf{sense\_closed} \;|\; X_t = \mathbf{is\_open})\;\;&=\;\;0.4\\
    p(Z_t = \mathbf{sense\_open} \;|\; X_t = \mathbf{is\_closed})\;\;&=\;\;0.2\\
    p(Z_t = \mathbf{sense\_closed} \;|\; X_t = \mathbf{is\_closed})\;\;&=\;\;0.8\\
\end{align*}
Therfore, my helper function for the sensor is the following.
\lstinputlisting[language=python, firstline=27, lastline=34]{hw8_bayes_filter.py}
\vspace{0.15in}
I used the condition on the right side of the probability statements to write the
if-else logic branching in the helper function. For the sake of making sure that
I only return a valid value from known strings, which are \lstinline{'open'} and
\lstinline{'closed'}, I returned $0.0$ when any other string is retrieved.\\
\vspace{0in}\\
My next helper function is used for retrieving values associate beliefs with
actions, which are given in the textbook/slides as the following.
\begin{align*}
    p(X_t = \mathbf{is\_open} \;|\; U_t = \mathbf{push},  X_{t\_1} = \mathbf{is\_open})\;\;&=\;\;1\\
    p(X_t = \mathbf{is\_closed} \;|\; U_t = \mathbf{push},  X_{t\_1} = \mathbf{is\_open})\;\;&=\;\;0\\
    p(X_t = \mathbf{is\_open} \;|\; U_t = \mathbf{push},  X_{t\_1} = \mathbf{is\_closed})\;\;&=\;\;0.8\\
    p(X_t = \mathbf{is\_closed} \;|\; U_t = \mathbf{push},  X_{t\_1} = \mathbf{is\_closed})\;\;&=\;\;0.2\\
    p(X_t = \mathbf{is\_open} \;|\; U_t = \mathbf{do\_nothing},  X_{t\_1} = \mathbf{is\_open})\;\;&=\;\;1\\
    p(X_t = \mathbf{is\_closed} \;|\; U_t = \mathbf{do\_nothing},  X_{t\_1} = \mathbf{is\_open})\;\;&=\;\;0\\
    p(X_t = \mathbf{is\_open} \;|\; U_t = \mathbf{do\_nothing},  X_{t\_1} = \mathbf{is\_closed})\;\;&=\;\;0\\
    p(X_t = \mathbf{is\_closed} \;|\; U_t = \mathbf{do\_nothing},  X_{t\_1} = \mathbf{is\_closed})\;\;&=\;\;1
\end{align*}
These are represented in my code as the following.
\lstinputlisting[language=python, firstline=36, lastline=53]{hw8_bayes_filter.py}
\vspace{0.15in}
Next, I defined a data-only class called \lstinline{FilterIteration} to represent
the filter iterations in a coherent manner.
The data members of this class are
simply \lstinline{self.action} and \lstinline{self.measurement}. In my main function,
I defined a list of instances of this \lstinline{FilterIteration} class, as given in
the prompt of this assignment. The list is shown below here.
\lstinputlisting[language=python, firstline=102, lastline=108]{hw8_bayes_filter.py}
\vspace{0.15in}
I now began implementating my Bayes Filter algorithm. The algorithm is contained
in the \lstinline{bayes_filter_algorithm()} function in my code. The code loops over
all states of the door and performs the prediction step and correction step for
each step. After these steps are done, the algorithm computes the normalizer,
$\eta$, and updates the \lstinline{base_belief_list} for usage in the next iteration.\\
\vspace{0in}\\
The prediction and correction step in my algorithm make use of helper functions,
named \lstinline{bf_predict_step()} and \lstinline{bf_correct_step()} respectively.
The prediction step helper function loops over all states and sums the products 
of the action-belief and base beliefs, as given in the pseudo-code algorithm in
the lecture slides. The correction step helper function simply returns the product
of the measurement belief and the prediction step result. This is also implemented
as seen in the slides. The code snippet for my implementation of the algorithm is
shown below here.
\lstinputlisting[language=python, firstline=61, lastline=97]{hw8_bayes_filter.py}
\vspace{0.15in}
Finally, after we run our iterations through this algorithm, we get the following
output.
\lstinputlisting{hw8_output.txt}

\end{document}