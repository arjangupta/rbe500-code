\documentclass[12pt]{article}
\usepackage{lipsum} %This package just generates Lorem Ipsum filler text. 
\usepackage{fancyhdr}
\usepackage[framed,numbered,autolinebreaks,useliterate]{mcode}
\usepackage{listings}
\usepackage{amsmath}
\usepackage{amsthm}
\usepackage{amsfonts}
\usepackage{tikz}
\usepackage[plain]{algorithm}

\newcommand{\hmwkTitle}{Homework\ \#8}
\newcommand{\hmwkClass}{RBE 500}
\newcommand{\hmwkAuthorName}{\textbf{Arjan Gupta}}

% Document settings

\topmargin=-0.45in
\evensidemargin=0in
\oddsidemargin=0in
\textwidth=6.5in
\textheight=9.0in
\headsep=0.25in

\linespread{1.1}

\pagestyle{fancy}
\lhead{\hmwkAuthorName}
\chead{\hmwkClass\ \hmwkTitle}
\cfoot{\thepage}

\renewcommand\headrulewidth{0.4pt}
\renewcommand\footrulewidth{0.4pt}

\setlength\parindent{0pt}
\setlength{\headheight}{15pt}


% Title Page
\title{
    \vspace{2in}
    \textmd{\textbf{\hmwkClass\ \hmwkTitle}}\\
    \vspace{3in}
}
\author{\hmwkAuthorName}
\date{}

% Function for citations
\newcommand{\supercite}[1]{~{\textsuperscript{\cite{#1}}}}

\begin{document}

\maketitle

\pagebreak

\begin{center}
    \large\textbf{Bayes Filter Report}
\end{center}
I started my Bayes Filter implementation in a python file called
\lstinline{hw8_bayes_filter.py}. At the top of the file, I added a \lstinline{VERBOSE}
flag that helps me toggle on and off output that is helpful for debugging, but
probably superfluous for the grader. I defined a list of states of the door,
which are \textit{open} and \textit{closed}. I also defined the list of 
`base' beliefs, which
are given in the slides/textbook as the following.
\begin{align*}
    bel(X_0 = \mathbf{open})\;\; &=\;\; 0.5\\
    bel(X_0 = \mathbf{closed})\;\; &=\;\; 0.5
\end{align*}
These are the initial beliefs (as denoted by $X_0$), which means that these values
will adjust as the script runs through the iteration cases. In my code, these are
simply a python list, followed by two helper functions. One helper function helps
us print the current belief values.
The other helper function makes sure our states, which are strings, are `mapped'
to the belief values, and returns them as such.
\lstinputlisting[language=python, firstline=8, lastline=25]{hw8_bayes_filter.py}
\vspace{0.1in}
Next, we start defining the helper function to get the belief values associated
with the sensor, i.e. the measurement beliefs. In the slides, these are given
as the following.
\lstinputlisting{hw8_output.txt}

\end{document}