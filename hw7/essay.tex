\documentclass[12pt]{article}
\usepackage{lipsum} %This package just generates Lorem Ipsum filler text. 
\usepackage{fancyhdr}

\newcommand{\hmwkTitle}{Homework\ \#7}
\newcommand{\hmwkClass}{RBE 500}
\newcommand{\hmwkAuthorName}{\textbf{Arjan Gupta}}

% Document settings

\topmargin=-0.45in
\evensidemargin=0in
\oddsidemargin=0in
\textwidth=6.5in
\textheight=9.0in
\headsep=0.25in

\linespread{1.1}

\pagestyle{fancy}
\lhead{\hmwkAuthorName}
\chead{\hmwkClass\ \hmwkTitle}
\cfoot{\thepage}

\renewcommand\headrulewidth{0.4pt}
\renewcommand\footrulewidth{0.4pt}

\setlength\parindent{0pt}
\setlength{\headheight}{15pt}


% Title Page
\title{
    \vspace{2in}
    \textmd{\textbf{\hmwkClass\ \hmwkTitle}}\\
    \vspace{3in}
}
\author{\hmwkAuthorName}
\date{}

% Function for citations
\newcommand{\supercite}[1]{~{\textsuperscript{\cite{#1}}}}

\begin{document}
\maketitle

\pagebreak

\begin{center}
    \large\textbf{Discussion Essay}
\end{center}

\textit{Please write a two-page essay (excluding references) discussing the existing and potential 
ethical issues in the application domain(s) of robotics you are interested in exploring in your 
graduate study and future career.}\\
\vspace{0in}\\
In this essay, I will be exploring the ethical concerns surrounding the advent of autonomous vehicles.
Specifically, I will focus on self-driving vehicles in the civilian space, such as cars, buses, and trucks.
As for a geographical and cultural focus, this essay will mostly take on a perspective encapsulated by 
transportation and traffic sensibilities within the United States of America.\\
\vspace{0in}\\
The V. Müller report states that a problem qualifies as a problem for AI/robot ethics if
we do \textit{not} readily know what the right thing to do is, further causing us to 
not readily know the answers to certain questions.\supercite{sep-ethics-ai}
There are several such unanswered questions with regard to level 5, 
full driving automation cars, as 
defined by the Society of Automotive Engineers.\supercite{SAE-levels}
Such vehicles will not even have steering wheels or 
acceleration/breaking pedals!\supercite{synopsys-web-article} Let us ponder on that for 
a moment. I ask the reader to imagine themselves in a car that just moves, without the 
ability to override any controls. Personally, I am a big fan of cruise-control and 
steering assistance in my own car, 
but I am not sure I would feel comfortable sitting in a car without 
the ability to override its controls. While this feels like a small 
thought experiment, it is 
in fact a question of ethics. Acceptance of level 5 autonomous cars requires humans 
to `trust' a machine, and just how averse will human society be towards this? Level 5
autonomy might become technologically possible, but will humans actually want to use
it?\\
\vspace{0in}\\
A survery conducted by Policygenius found that 76\% of Americans feel less
safe driving or riding in cars with self-driving features.\supercite{policy-genius-survey}
This seems like an astonishingly high number, but is perhaps understandable.
Before any major technological advancement, humans have usually felt skeptical
of the technology in question. Most humans likely felt unsafe traveling
in commercial aircraft when it first came into existence. Yet, if we glance at all the
ten debate topics outlined in the V. Müller report, only a few could explain 
this general insecurity around AVs (autonomous vehicles). After, besides a few
corner-cases of morality (for example, the famous trolley problem), AVs are
supposed to be designed to follow traffic rules, which intrinsically have 
very few `grey areas'. One of these ten debate topics, however, appears
as a general logical explanation --- the opacity of AI systems.\supercite{sep-ethics-ai}
Here, it is explained that the AI systems behind AVs are generally invisible
to both the user and the programmer. It is therefore difficult to trust a `black-box'
system to take charge of moving heavy machinery that could endanger human life.\\
\vspace{0in}\\
The antidote to this distrust seems to be multifaceted, however two major facets
are widely available statistics on the safety of AVs, and technical education of the public.
A well-cited, comprehensive article from the AI Ethics journal lists
among its many conclusions that AVs have the potential to increase traffic safety,
and that people with high education levels are more positive towards AVs
than people with lower education levels.\supercite{Othman2021-xr} Therefore,
as society begins to see more widespread studies and data on the safety of AVs,
people might begin to trust AVs more. Furthermore, as education systems begin
to focus more on STEM, the general understanding of AI will begin to increase.\\
\vspace{0in}\\
An additional perspective is that the opacity of AI systems could eventually
decrease its scope from being opaque to both programmer and user to just the user.
For example, currently, most programmers view neural networks (NNs) as a black box. We
understand that NNs have input, output, and middle layers, but the fact that the middle
layers re-adjust their weights via backpropagation makes it extremely difficult to
know why a fully-trained NN makes a prediction in a certain way. However a recent
manuscript by Caglar Aytekin has shown that any neural network with any activation
function can be represented as a decision tree.\supercite{caglar-nn-dt} Studies 
like this will give humans a deeper understanding of AI, and can evetually help
users of autonomous vehicles battle against the `fear of the unknown'.\\
\vspace{0in}\\
\lipsum[7]\\
\vspace{0in}\\
\lipsum[10]

\pagebreak

\bibliography{refs}
\bibliographystyle{plain}

\end{document}