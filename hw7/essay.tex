\documentclass[12pt]{article}
\usepackage{lipsum} %This package just generates Lorem Ipsum filler text. 
\usepackage{fancyhdr}

\newcommand{\hmwkTitle}{Homework\ \#7}
\newcommand{\hmwkClass}{RBE 500}
\newcommand{\hmwkAuthorName}{\textbf{Arjan Gupta}}

% Document settings

\topmargin=-0.45in
\evensidemargin=0in
\oddsidemargin=0in
\textwidth=6.5in
\textheight=9.0in
\headsep=0.25in

\linespread{1.1}

\pagestyle{fancy}
\lhead{\hmwkAuthorName}
\chead{\hmwkClass\ \hmwkTitle}
\cfoot{\thepage}

\renewcommand\headrulewidth{0.4pt}
\renewcommand\footrulewidth{0.4pt}

\setlength\parindent{0pt}
\setlength{\headheight}{15pt}


% Title Page
\title{
    \vspace{2in}
    \textmd{\textbf{\hmwkClass\ \hmwkTitle}}\\
    \vspace{3in}
}
\author{\hmwkAuthorName}
\date{}

% Function for citations
\newcommand{\supercite}[1]{~{\textsuperscript{\cite{#1}}}}

\begin{document}
\maketitle

\pagebreak

\begin{center}
    \large\textbf{Discussion Essay}
\end{center}

\textit{Please write a two-page essay (excluding references) discussing the existing and potential 
ethical issues in the application domain(s) of robotics you are interested in exploring in your 
graduate study and future career.}\\
\vspace{0in}\\
In this essay, I will be exploring the ethical concerns surrounding the advent of autonomous vehicles.
Specifically, I will focus on self-driving vehicles in the civilian space, such as cars, buses, and trucks.
As for a geographical and cultural focus, this essay will mostly take on a perspective encapsulated by 
transportation and traffic sensibilities within the United States of America.\\
\vspace{0in}\\
The V. Müller report states that a problem qualifies as a problem for AI/robot ethics if
we do \textit{not} readily know what the right thing to do is, causing us to 
not readily know the answers to certain questions.\supercite{sep-ethics-ai}
There are several such unanswered questions with regards to level 5, 
full driving automation\supercite{SAE-levels} cars, as 
defined by the Society of Automotive Engineers. Such vehicles will not even have 
steering wheels or 
acceleration/breaking pedals!\supercite{synopsys-web-article} Let us ponder on that for 
a moment. I ask the reader to imagine themselves in a car that just moves, without the 
ability to override any controls. Personally, I am a big fan of cruise-control and 
steering assistance in my own car, 
but I am not sure I would feel comfortable sitting in a car without 
the ability to override its controls. While this feels like a small 
thought experiment, it is 
in fact a question of ethics. Acceptance of level 5 autonomous cars requires humans 
to `trust' a machine, and just how averse will human society be towards this? Level 5
autonomy might become technologically possible, but will humans actually want to use
it?
\vspace{0in}\\
\lipsum[3]\\
\vspace{0in}\\
\lipsum[4]\\
\vspace{0in}\\
\lipsum[5]\\
\vspace{0in}\\
\lipsum[6]\\
\vspace{0in}\\
\lipsum[7]\\
\vspace{0in}\\
\lipsum[8]\\
\vspace{0in}\\
\lipsum[9]\\
\vspace{0in}\\
\lipsum[10]

\bibliography{refs}
\bibliographystyle{plain}

\end{document}